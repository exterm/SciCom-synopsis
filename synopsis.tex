\documentclass[a4paper, 12pt]{article}

\usepackage[margin=2cm,top=2.7cm,bottom=2.7cm]{geometry}
\usepackage{german}
\usepackage[utf8]{inputenc}
\usepackage{lastpage}
\usepackage[bookmarks, hidelinks]{hyperref}
\usepackage{fancyhdr}
\usepackage{setspace}
\usepackage{comment}
\usepackage{graphicx}
\usepackage{calc}
\usepackage{ifthen}
\usepackage{boxedminipage}
\usepackage{amsfonts}
\usepackage{amsmath}
\usepackage{enumitem}

\sloppy
\setlength{\parindent}{8pt}
\setlist[description]{itemindent=-30pt, leftmargin=50pt, itemsep=.5em}

\pagestyle{fancy}
 
\lhead{Scientific Computing}
\chead{Zusammenfassung}
\rhead{WS 2011/2012}

\lfoot{Philip Müller, inf9293}
\cfoot[Seite \thepage\ von \pageref{LastPage}]{Seite \thepage\ von \pageref{LastPage}}
\rfoot{\today}
\renewcommand{\footrulewidth}{.5pt}

%\setcounter{tocdepth}{2}

\begin{document}

\tableofcontents
\pagebreak



\part{Grundlagen}



\section{Definition}
Scientific Computing



\section{Numerik-Grundlagen}


\subsection{Reelle Zahlen}


\subsection{Komplexe Zahlen}


\subsection{Matrizen}


\subsection{Vektoren}



\section{Nichtlineare Gleichungen}


\subsection{Bisektionsverfahren}


\subsection{Newton-Verfahren}


\subsection{Fixpunkt-Iteration}


\subsection{Algebraische Polynome}


\subsection{Newton-Horner-Verfahren}



\section{Numerisches Differenzieren}


\subsection{Taylor-Entwicklung}



\section{Numerisches Integrieren}


\subsection{Problem}


\subsection{Mittelpunktformel}


\subsection{Trapezformel}


\subsection{Simpson-Formel}




\part{Methoden}




\part{Parallelisierung}


%%%%%%%%%%



\section{Bli}


\subsection{bla}

\subsubsection*{blubb}

\begin{itemize}
  \item allgemeine Eigenschaften
\end{itemize}
\begin{itemize}
  \renewcommand{\labelitemi}{\(-\)}%
  \item Nachteile
\end{itemize}
\begin{itemize}
  \renewcommand{\labelitemi}{+}%
  \item Vorteile
\end{itemize}



\end{document}
