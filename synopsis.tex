\documentclass[a4paper, 12pt]{article}

\usepackage[margin=2cm,top=2.7cm,bottom=2.7cm]{geometry}
\usepackage{german}
\usepackage[utf8]{inputenc}
\usepackage{lastpage}
\usepackage[bookmarks, hidelinks]{hyperref}
\usepackage{fancyhdr}
\usepackage{setspace}
\usepackage{comment}
\usepackage{graphicx}
\usepackage{calc}
\usepackage{ifthen}
\usepackage{boxedminipage}
\usepackage{amsfonts}
\usepackage{amsmath}
\usepackage{enumitem}

\sloppy
\setlength{\parindent}{8pt}
\setlist[description]{itemindent=-30pt, leftmargin=50pt, itemsep=.5em}

\pagestyle{fancy}
 
\lhead{Scientific Computing}
\chead{Zusammenfassung}
\rhead{WS 2011/2012}

\lfoot{Philip Müller, inf9293}
\cfoot[Seite \thepage\ von \pageref{LastPage}]{Seite \thepage\ von \pageref{LastPage}}
\rfoot{\today}
\renewcommand{\footrulewidth}{.5pt}

%\setcounter{tocdepth}{2}

\begin{document}

\tableofcontents
\pagebreak



\part{Grundlagen}



\section{Definition}
Scientific Computing ist
\begin{itemize}
  \item numerische Simulation
  \item Lösung numerischer Probleme aus verschiedenen wissenschaftlichen Disziplinen
\end{itemize}



\section{Numerik-Grundlagen}


\subsection{Reelle Zahlen}

\subsubsection*{Repräsentation}
\begin{itemize}
  \item Im Rechner üblicherweise Teilmenge der reellen Zahlen benutzt: Gleitkommazahlen
  \item Gleitkomma-Repräsentation einer reellen Zahl verlustbehaftet, ``beschnittene Zahl''
\end{itemize}
\begin{description}
  \item[Darstellung] durch
    \begin{itemize}
      \item Vorzeichen
      \item Basis
      \item Mantisse
      \item Exponent
    \end{itemize}
  \item[Rundungsfehler] beim Repräsentieren
    \begin{itemize}
      \item \(|x-fl(x)|/|x| \le 1/2 \epsilon_M\)
      \item \(\epsilon_M = \beta^{1-t}\) gibt Abstand zwischen 1 und der nächsten Gleitkommazahl größer 1 an
      \item \(\epsilon_M\) in MATLAB durch das Kommando \texttt{eps} bestimmbar
    \end{itemize}
\end{description}

\subsubsection*{Eingeschränkte Gesetze}
\begin{itemize}
  \item Kommutativität gilt
  \item die 0 ist nicht eindeutig
  \item Assoziativität und Distributivität gelten nicht!\\
    z.B. bei Rechnungen die an die Ränder des Zahlenbereichs gehen
\end{itemize}


\subsection{Komplexe Zahlen}
--- hier Text einfügen ---


\subsection{Matrizen}
--- hier Text einfügen ---


\subsection{Vektoren}
--- hier Text einfügen ---




\part{Methoden}



\section{Nichtlineare Gleichungen}


\subsection{Bisektionsverfahren}
--- hier Text einfügen ---


\subsection{Newton-Verfahren}
--- hier Text einfügen ---


\subsection{Fixpunkt-Iteration}
--- hier Text einfügen ---


\subsection{Algebraische Polynome}
--- hier Text einfügen ---


\subsection{Newton-Horner-Verfahren}
--- hier Text einfügen ---



\section{Numerisches Differenzieren}


\subsection{Taylor-Entwicklung}
--- hier Text einfügen ---



\section{Numerisches Integrieren}


\subsection{Problem}
--- hier Text einfügen ---


\subsection{Mittelpunktformel}
--- hier Text einfügen ---


\subsection{Trapezformel}
--- hier Text einfügen ---


\subsection{Simpson-Formel}
--- hier Text einfügen ---



\section{Lineare Systeme}


\subsection{LU-Faktorisierung}
--- hier Text einfügen ---


\subsection{Gauß-Faktorisierung}
--- hier Text einfügen ---


\subsection{Pivoting}
--- hier Text einfügen ---



\section{Gewöhnliche Differentialgleichungen}


\subsection{Allgemeines}
--- hier Text einfügen ---


\subsection{Euler-Verfahren}
--- hier Text einfügen ---


\subsection{Verfahren höherer Ordnung}
--- hier Text einfügen ---



\section{Interpolation}


\subsection{Langrange-Interpolation}
--- hier Text einfügen ---


\subsection{Runge-Interpolation}
--- hier Text einfügen ---


\subsection{Chebyshev-Interpolation}
--- hier Text einfügen ---


\subsection{Stückweise Lineare Interpolation}
--- hier Text einfügen ---


\subsection{Spline-Funktionen}
--- hier Text einfügen ---



\section{Eigenwerte und Eigenvektoren}


\subsection{Potenzverfahren}
--- hier Text einfügen ---


\subsection{QR-Zerlegung}
--- hier Text einfügen ---




\part{Parallelisierung}



\section{IPython}



\section{Leistungsmaße}


\subsection{Laufzeit}
--- hier Text einfügen ---


\subsection{Speedup}
--- hier Text einfügen ---



\section{Zerlegungsmethoden}


\subsection{Allgemeines}
--- hier Text einfügen ---


\subsection{Funktionale Zerlegung}
--- hier Text einfügen ---


\subsection{Datenzerlegung}
--- hier Text einfügen ---


\subsection{Funktions- und Datenzerlegung}
--- hier Text einfügen ---



\section{Beispiele}


\subsection{Parallele LU-Faktorisierung}
--- hier Text einfügen ---






%%%%%%%%%%



\section{Bli}


\subsection{bla}

\subsubsection*{blubb}

\begin{itemize}
  \item allgemeine Eigenschaften
\end{itemize}
\begin{itemize}
  \renewcommand{\labelitemi}{\(-\)}%
  \item Nachteile
\end{itemize}
\begin{itemize}
  \renewcommand{\labelitemi}{+}%
  \item Vorteile
\end{itemize}



\end{document}
